% \documentclass[10pt, conference, compsocconf]{IEEEtran}
%\documentclass[conference]{IEEEtran}
\documentclass[letter]{sig-alternate}
\usepackage{eso-pic,xcolor}
\makeatletter
%
\pdfpagewidth=8.5in
\pdfpageheight=11in
\usepackage{flushend}
%\AddToShipoutPicture*{%
%\setlength{\@tempdimb}{20pt}%
%\setlength{\@tempdimc}{\paperheight}%
%\setlength{\unitlength}{1pt}%
%\put(\strip@pt\@tempdimb,\strip@pt\@tempdimc){%
%   \makebox(0,-60)[l]{\color{blue}%
%
%  }%
  
%\put(\strip@pt\@tempdimb,\strip@pt\@tempdimc){%
%    \makebox(0,-85)[l]{\color{black}%
%}%
%}%

%\put(\strip@pt\@tempdimb,\strip@pt\@tempdimc){%
%    \makebox(0,-110)[l]{\color{black}%
%}%
%  }%
%}

%
\def\ps@headings{%
\def\@oddhead{\mbox{}\scriptsize\rightmark \hfil \thepage}%
\def\@evenhead{\scriptsize\thepage \hfil \leftmark\mbox{}}%
\def\@oddfoot{}%
\def\@evenfoot{}}
\makeatother




\pagestyle{headings}

\title{Next-NDNVideo: live and pre-recorded streaming \\\ using Consumer / Producer API over NDN}

\author{
%Paper \#231, 13 pages
Lijing Wang \\ {\normalsize Tsinghua University} \\ {\normalsize wanglj11@mails.tsinghua.edu.cn }
\and Ilya Moiseenko \\ {\normalsize UCLA} \\ {\normalsize iliamo@cs.ucla.edu }
\and Lixia Zhang \\ {\normalsize UCLA} \\ {\normalsize lixia@cs.ucla.edu }
}


\newfont{\nicettfont}{cmtt10}
\newcommand{\ndnName}[1]{``{\nicettfont #1}''}
\renewcommand{\texttt}[1]{{\nicettfont #1}}

\newcommand{\todo}[1]{\vspace{2 mm}\par \noindent \marginpar{\textsc{ToDo}}
\framebox{\begin{minipage}[c]{0.95 \columnwidth}
\tt #1 \end{minipage}}\vspace{5 mm}\par}

\usepackage{graphicx}
% \usepackage[colorlinks]{hyperref}
\usepackage[]{hyperref}
\usepackage{breakurl}
\usepackage{url}
\usepackage[nocompress]{cite}
\usepackage{amsmath}
% \usepackage{verbatim}
% \usepackage{algpseudocode}
\usepackage{algpseudocode,algorithm}
% More customizeable version. Probably it would be better to convert pseudocode to 2e format
% \usepackage{algorithm2e} 
\usepackage{multirow}
\usepackage{times}
\usepackage{color}

\graphicspath{{figures/}}

\begin{document}

\maketitle

% \begin{abstract}
% Named Data Networking (NDN) is a general purpose protocol offering rich functionality at the network layer: caching, multi-path forwarding, multicast delivery, and data-based security model. Above the network layer, system libraries simplify application developers' tasks by providing an easy to use yet powerful API to utilize the functions enabled by NDN.  This paper presents the design of a Consumer / Producer programming interface that supports application level framing via new data retrieval protocols, and several supporting mechanisms to make NDN application programming easier and faster.
% \end{abstract}


%\begin{IEEEkeywords}
%Information-centric networks, named data networking, API, inter-process communication
%\end{IEEEkeywords}
\input{01introduction}
%!TEX root = nextndnvideo-tr.tex
\section{background} % (fold)
\label{sec:background}
% section section_name (end)
%!TEX root = nextndnvideo-tr.tex
\vspace{0.3cm}
\section{design} % (fold)
\label{sec:design}
\subsection{Architecture}
Next-NDNVideo is based on Consumer / Producer API over Named Data Networking. It contains two kinds of roles - producer and consumer. The whole architecture (Figure~\ref{fig:arch}) is described below.

\begin{figure*}%[htbp]
  \centering
  \includegraphics[scale=0.55]{architecture}
  % \vspace{-0.3cm}
  \caption{Architecture of Next-NDNVideo}
  \label{fig:arch}
  %\vspace{-0.2cm}
\end{figure*}

The producer is responsible for generating video and audio streaming. It behaves as the content publisher. We use Gstreamer to extract the video and audio frames from video source. Then we call Consumer / Producer API to encapsulate the data frames into NDN packets. In this case these NDN packets are available to fetch. The producer part can work even without being attached to the NDN network. The producer's packages will be cached in the \textit{Send Buffer} the API maintained and \textit{Content Store} of NFD\cite{nfd-guide}. Also the NDN packets can be written into Repo\cite{repo-ng}. Then Repo will take charge of satisfying the data request.

Every time the consumer wants to play back some video, it should send interests to fetch the data by using one of the Data Retrieval Protocol \textit{(SDR/UDR/RDR)}. When Consumer / Producer API brings data back, the call back function will be triggered to retrieve data. The reassembled video or audio frame will be passed to Gstreamer for further processing. Finally, Gstreamer will provide the decoded data to video player for playing back. Because all NDN applications are interest-driven, only the consumer keeps sending interest, it can fetch the data and play it back.

According to the content generating and data retrieval pattern, Next-NDNVideo can be divided into two different implementations. One is Live Streaming, which the producer captures video from camera and audio from microphone and keeps publishing them as a live stream. Any time the consumer sends interest asking for the video stream, it will get the latest video and audio. Another is Pre-recorded Streaming, which is more like youtube. In this case the consumer can send interest asking for the latest playing list and chose one to play. The video and audio frames associated with one video file will be written into Repo in advance. The producer only needs to keep publishing the latest playing list containing all the names of video files that are ready to play.

\subsection{Naming Structure}
Next-NDNVideo produces video and audio stream separately. Every single frame will form a piece of data, so they need a unique name. And before consuming the video and audio content, it should first use the stream information to set up the playing pipeline. The following is an example name of Pre-recorded Streaming. 
\begin{quote}
``/ndn/ucla/recordvideo/video-1234/video/content \\\ /8/\%00\%00''
\end{quote}
\begin{itemize}
	\item{\textbf{Routing Prefix:}} ``/ndn/ucla/recordvideo'' is the prefix.
	\item{\textbf{Video Name:}} ``/video-1234'' is a representation for one specific video such as file name.
	\item{\textbf{Video Mark:}} ``/video'' is a mark to distinguish video and audio.
	\item{\textbf{Streaminfo Mark:}} ``/content'' represents the frames and ``/streaminfo'' represents the stream information.
	\item{\textbf{Frame Number:}} ``/8'' is frame number, which Streaminfo does not have this component.
	\item{\textbf{Segment Number:}} ``\%00\%00'' is the segment number. Because most video frames would contain more than one segment, this component is essential. As we mentioned before the Consumer / Producer API will do the segmentation processing, so the segment number will be appended by the API automatically. Streaminfo does not have this component, neither.
\end{itemize}

Then we can conclude that the above name stands for a piece of data which is the segment 0 inside the 8th video frame of video-1234 under the prefix of /come/youtube. The relative stream information name is as following.
\begin{quote}
``ndn/ucla/recordvideo/video-1234/video/streaminfo \\\ /pipeline''
\end{quote}

Except for the last component, the others are all introduced above. The last component of Streaminfo is the information type.
\begin{itemize}
	\item{\textbf{Info Type:}} 
	\begin{itemize}
		\item{\textbf{pipeline:}} means that it asks for essential information to set up the playing back pipeline. 
		\item{\textbf{final\_id:}} means that it asks for the final frame number of this video. 
	\end{itemize}
\end{itemize}


% \begin{figure}%[htbp]
%   \centering
%   \includegraphics[scale=0.5]{listnaming}
%   % \vspace{-0.3cm}
%   \caption{NDN packet types.}
%   \label{fig:listnaming}
% \end{figure}

% section section_name (end)
\input{04implementation}
%!TEX root = nextndnvideo-tr.tex
\section{evaluation} % (fold)
\label{sec:evaluation}
When some frames missing, the performance does not be affected too much. The audio will not be affected at all. For video, when the missing video frame is the key frame, it will appear one second mosaic. But if it was other type frame, the picture is fluent enough.
How to measure the quantification?

% section evaluation (end)
%!TEX root = nextndnvideo-tr.tex
\section{comparison to prior NDNVideo} % (fold)
\label{sec:comparison}
The main difference between Next-NDNVideo and NDNVideo is the way how we handle framing. In the prior NDNVideo project, the video or audio stream is chopped into fixed size. The segment can just fill in one NDN package. A mapping between time and segment number is introduced to keep the video and audio synced. The seeking is also supported by the time-segment mapping mechanism. 

In our project, the video and audio is chopped into frames. One frame may contain several segments. The segmentation process is handled by Consumer / Producer API. The application only focuses on the frame level and leave other task to Consumer / Producer API. We think this application level framing is more like the true NDN way, which we mentioned in chapter 2\ref{sec:background}. Every frame has a unique name and is produced and consumed in one time. Only one frame missing won't affect other frames, thus leverage the whole impact to the playing back. 

On the contrary, the fixed size segmentation breaks the integrity of frames. Only when all the packages are received correctly, the playing back progress can be guaranteed.  So we think the prior NDNVideo is more like a TCP way, but not NDN.  The application level framing also provides the flexibility to the video consumer. For example, in the live streaming case, if the previous frame can't be retrieved on time or not integrated, the consumer can just skip this bad frame to keep the video streaming. I can see that it won't influence the video fluency according to our evaluation. Table one shows other differences such as dependencies, Gstreamer version and coding language.

[table needed here]
% section comparison (end)
%!TEX root = nextndnvideo-tr.tex
\section{conclusion} % (fold)
\label{sec:conclusion}
% section section_name (end)

%\input{10-acknowledgement}

\bibliographystyle{IEEEtran}
\bibliography{references}


\end{document}
