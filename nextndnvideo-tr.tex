% \documentclass[10pt, conference, compsocconf]{IEEEtran}
%\documentclass[conference]{IEEEtran}
\documentclass{sig-alternate}
\usepackage{eso-pic,xcolor}
\makeatletter
%
\pdfpagewidth=8.5in
\pdfpageheight=11in
\AddToShipoutPicture*{%
\setlength{\@tempdimb}{20pt}%
\setlength{\@tempdimc}{\paperheight}%
\setlength{\unitlength}{1pt}%
\put(\strip@pt\@tempdimb,\strip@pt\@tempdimc){%
    \makebox(0,-85)[l]{\color{blue}%
NDN, Technical Report NDN-0031, 2015. \url{http://named-data.net/techreports.html}}
  }%
\put(\strip@pt\@tempdimb,\strip@pt\@tempdimc){%
    \makebox(0,-105)[l]{\color{black}%
Revision 1: April 30, 2015}
  }%
}

%
\def\ps@headings{%
\def\@oddhead{\mbox{}\scriptsize\rightmark \hfil \thepage}%
\def\@evenhead{\scriptsize\thepage \hfil \leftmark\mbox{}}%
\def\@oddfoot{}%
\def\@evenfoot{}}
\makeatother




\pagestyle{headings}

\title{NDNlive and NDNtube: Live and Pre-recorded \\\ Video Streaming over NDN}

\author{
%Paper \#231, 13 pages
Lijing Wang \\ {\normalsize Tsinghua University} \\ {\normalsize wanglj11@mails.tsinghua.edu.cn }
\and Ilya Moiseenko \\ {\normalsize UCLA} \\ {\normalsize iliamo@cs.ucla.edu }
\and Lixia Zhang \\ {\normalsize UCLA} \\ {\normalsize lixia@cs.ucla.edu }
}


\newfont{\nicettfont}{cmtt10}
\newcommand{\ndnName}[1]{``{\nicettfont #1}''}
\renewcommand{\texttt}[1]{{\nicettfont #1}}

\newcommand{\todo}[1]{\vspace{2 mm}\par \noindent \marginpar{\textsc{ToDo}}
\framebox{\begin{minipage}[c]{0.95 \columnwidth}
\tt #1 \end{minipage}}\vspace{5 mm}\par}

\usepackage{graphicx}
% \usepackage[colorlinks]{hyperref}
\usepackage[]{hyperref}
%\usepackage{breakurl}
\usepackage{url}
\usepackage[nocompress]{cite}
\usepackage{amsmath}
% \usepackage{verbatim}
% \usepackage{algpseudocode}
\usepackage{algpseudocode,algorithm}
% More customizeable version. Probably it would be better to convert pseudocode to 2e format
% \usepackage{algorithm2e} 
\usepackage{multirow}
\usepackage{times}
\usepackage{color}
\paperheight=11in
\graphicspath{{figures/}}

\begin{document}

\maketitle

\begin{abstract}
Named Data Networking offers significant advantages for media distribution applications: in-network caching and multicast capabilities. This technical report provides a detailed view on two video streaming applications: NDNlive and NDNtube. 
NDNlive is capable of streaming live video captured by the camera and handling network problems by dropping individual video or audio frames. NDNtube prototypes Youtube-like user experience by serving dynamically generated playlist and streaming the media in its original quality. Both applications were developed on top of Consumer / Producer API providing a convenient way of publishing and fetching ADUs of any size, and Gstreamer library providing media decoding/encoding functionality.
%NDNlive is a live video streaming application built on top of UDR protocol of Consumer / Producer API, and NDNtube is a pre-recorded video streaming application built on top of RDR protocol. Video and audio streams are converted into sequences of individual frames by the Gstreamer media processing library and are published independently. Individual frames are later retrieved by UDR and RDR protocols offered in Consumer / Producer API.
\end{abstract}


%\begin{IEEEkeywords}
%Information-centric networks, named data networking, API, inter-process communication
%\end{IEEEkeywords}
\input{01introduction}
%!TEX root = nextndnvideo-tr.tex
\section{background} % (fold)
\label{sec:background}
% section section_name (end)
%!TEX root = nextndnvideo-tr.tex

\begin{figure*}[htbp]
  \centering
  \includegraphics[scale=0.55]{NDNtube_arch}
  % \vspace{-0.3cm}
  \caption{NDNtube Architecture}
  \label{fig:NDNtube_arch}
  %\vspace{-0.2cm}
\end{figure*}

\section{design goals} % (fold)
\label{sec:design_goals}
UCLA REMAP lab put efforts in building NDNVideo~\cite{ndnvideo} that was one of the first showcases in support of the claim of the possibility of video streaming over NDN. The later changes in the packet format and the development of the new NDN forwarding daemon (NFD~\cite{nfd-guide}) made this implementation of video streaming software obsolete. 

NDNlive and NDNtube is the new effort of UCLA IRL lab to have similar functionality to NDNvideo compatible with the new packet format, new forwarder and new application libraries, such as Consumer / Producer API. The secondary goal of developing these application is to provide more complete examples for application developers learning the new Consumer / Producer API.

The following list contains the key features of NDNlive and NDNtube applications:
\begin{itemize}
\item{\textit{``Live and pre-recorded video\&audio streaming to multiple users''}}

NDNlive provides the live video\&audio streaming to multiple users and guarantees the fluency of the streaming. NDNtube prototypes a Youtube-like user experience: 1) selection of a media from the list of available media resources, and 2) smooth buffered playback. 

\item{\textit{``Random access based on actual location in the video''}}

The video stream is organized as two series of media frames: video and audio. The relationship between playback time and the frame number is non-ambiguous, because both applications use constant frame-rate encoding. In other words, the frame number can be easily calculated from the playback time information and the video and audio frame rate. 

\textbf{ }

\item{\textit{``Synchronized playback''}}

At the moment of data production, each frame is extracted in the form of \textit{GstBuffer}~\cite{Gstbuffer}, which contains playback timestamp information. When audio and video frames are retrieved at the consumer side, video and audio streams are synchronized by Gstreamer library automatically. Another way to achieve synchronization without Gstreamer library is to consider the direct relationship between frame-rate, playback time, and frame numbers.

\item{\textit{``Connection-less and session-less streaming''}}

NDNlive and NDNtube consumers do not try to establish any persistent session or connection with the video publisher. Video and audio frames are fetched from either from in-network cache, or producer's cache, or the permanent storage (e.g. Repo), when producer goes offline.

%\item{\textit{``Archival access to live streams''}}

%The live stream can also be written into the Repo. And Repo will take over the duty of Interest satisfaction. Then the archival access to live stream is possible.

\item{\textit{``Content verification and provenance''}}

Every Data packet must be signed with an asymmetric key of the original publisher in order to reliable authenticate content. Video publishing pipeline outputs so many Data packets that the security component of ndn-cxx library becomes a bottleneck, because of its limited signing speed. In order to achieve the desired signing throughput, both NDNlive and NDNtube producers use Producer API with FAST\_SIGNING option.

\end{itemize} 

% section section_name (end)
%!TEX root = nextndnvideo-tr.tex
\vspace{0.3cm}
\section{Design} % (fold)
\label{sec:arch}
NDNLive and NDNTube are all based on Consumer / Producer API over Named Data Networking. They contain two kinds of roles - producer and consumer. According to the content production and data retrieval pattern, their architecture and namespace will be described separately in this section. 

\paragraph{NDNLive Architecture} % (fold)
\vspace{0.3cm}
\label{par:ndnlive_arch}
The architecture of NDNLive is shown as (Figure~\ref{fig:ndnlive_arch}), and is illustrated as below:

\begin{itemize}
  \item \textit{Producer}

  NDNLive is \textit{Live Streaming}, which the producer captures video from camera and audio from microphone, then passes them to Gstreamer to get raw data encoded and extract the video and audio frames. At last the frames are published to NDN Network by Consumer / Producer API. 

  \item \textit{Consumer}

  The consumer can send interest asking for the video stream at any time, it will get the latest video and audio frames, then pass them to Gstreamer to get decoded and at the end the player can play them back. 

\end{itemize}



\paragraph{NDNTube Architecture} % (fold)
\vspace{0.3cm}
\label{par:ndntube_arch}
The architecture of NDNLive is shown as (Figure~\ref{fig:ndntube_arch}), and is illustrated as below:
\begin{itemize}
  \item \textit{Producer}

    NDNTube is \textit{Pre-recorded Streaming}, which is more like Youtube. The video source is the pre-recorded video file. As we described in Section~\ref{sec:repo}, the video and audio frames associated with this video file will be written into Repo in advance. And Repo will take over the duty of responding to the Interests requesting for frames. Then there is no need for the frame producer being attached to the NDN Network. 

  \item \textit{Consumer}

    Another difference from NDNLive, in this case the consumer must know what video files the producer has. So the consumer should send Interest asking for the latest playlist and then chose one to play. So the producer only needs to keep publishing the latest playlist containing all the names of video files to the NDN Network.

\end{itemize}


\subsection{Namespace}

NDNLive and NDNTube produce video and audio stream separately. Every single frame need a unique name. And before consuming the video and audio content, it should first use the stream information to set up the playing pipeline. There are many components in common between them. 

\paragraph{NDNLive Naming} % (fold)
\label{par:ndnlive_naming}
\vspace{0.3cm}
The following is an example name of NDNLive. 

\begin{quote}
``/ndn/ucla/ndnlive/stream-1/video/content/8/\%00\%00''
\end{quote}
\begin{itemize}
	\item{\textbf{Routing Prefix:}} ``/ndn/ucla/ndnlive'' is the prefix.
	\item{\textbf{Stream\_ID:}} ``/stream-1'' is a representation for one specific live stream, because there could be several producers under the same prefix to publish different live stream.
	\item{\textbf{Video or Audio:}} ``/video'' is a mark to distinguish video and audio.
	\item{\textbf{Content or Stream\_Info:}} ``/content'' represents the frames and ``/stream\_info'' represents the stream information.
	\item{\textbf{Frame Number:}} ``/8'' is frame number, which stream\_info does not have this component.
	\item{\textbf{Segment Number:}} ``\%00\%00'' is the segment number. Because most video frames would contain more than one segment, this component is essential. As we mentioned before (Section~\ref{ssub:cpapi}), the Consumer / Producer API will do the segmentation processing, so the segment number will be appended by the API automatically. But audio frame in NDNLive is always smaller than one segment. There is no segment number for audio frame, and stream\_info does not have this component, neither.
\end{itemize}

Then we can conclude that the above name stands for a piece of data which is the segment 0 inside the 8th video frame of stream-1 under the prefix of /ndn/ucla/ndnlive. 

The relative stream information name is shown as below:
\begin{quote}
``ndn/ucla/ndnlive/stream-1/video/stream\_info/ \\\ 1428725107049''
\end{quote}

Because the stream\_info would contain the current frame number of video and audio. And the consumer always want to retrieve the latest stream\_info to set up the pipeline and also the current frame number to start consuming from this number. So we append a timestamp component at the end of stream\_info to help the consumer retrieve the latest one.

The whole namespace of NDNLive should look like Figure~\ref{fig:ndnlive_naming}.

\begin{figure}%[htbp]
  \centering
  \includegraphics[scale=0.3]{ndnlive_naming}
  % \vspace{-0.3cm}
  \caption{NDNLive Namespace}
  \label{fig:ndnlive_naming}
  %\vspace{-0.2cm}
\end{figure}
% paragraph ndnlive_naming (end)
\paragraph{NDNTube Naming} % (fold)
\label{par:ndntube_naming}
\vspace{0.3cm}

The namespace of NDNTube is very similar to NDNLive. There are four differences:
\begin{enumerate}
	\item{\textit{Playlist added}} 
		
		NDNTube will have a playlist component, which NDNLive does not. The name example is shown as below: 
		\begin{quote}
		``/ndn/ucla/ndntube/playlist/1428725107042''
		\end{quote}
    Because the playlist can be changed at anytime as long as a new file is added or deleted. The consumer always wants to retrieve the latest one. We append a timestamp component at the end to distinguish the obsolete and latest playlist.

	\item{\textit{Video\_Name instead of Stream\_ID}} 

		We should set a component to specific one video file instead of a live stream\_id.

	\item{\textit{No Timestamp under Stream\_Info}} 

    In NDNTube the related stream information of one video file is fixed (basic information to set up the pipeline and the final frame number marked as the \textit{EOS} of the video file), so there is no need to add the timestamp component.

	\item{\textit{Audio also needs Segment\_Number}} 

    The audio frames may also contain more than one segments, because it's not under our control (We control the audio frame small enough for NDNLive), the quality of MP4 file will influence the size of audio frames.
\end{enumerate}

The whole namespace of NDNTube is shown as Figure~\ref{fig:ndntube_naming}.

\begin{figure}%[htbp]
  \centering
  \includegraphics[scale=0.3]{ndntube_naming}
  % \vspace{-0.3cm}
  \caption{NDNTube Namespace}
  \label{fig:ndntube_naming}
  %\vspace{-0.2cm}
\end{figure}
% paragraph ndntube_naming (end)

% section section_name (end)
%%% Local Variables:
%%% mode: latex
%%% TeX-master: "nextndnvideo-tr"
%%% End:

%!TEX root = nextndnvideo-tr.tex
\section{implementation} % (fold)
\label{sec:implementation}
NDNLive and NDNTue are both developed using Consumer / Producer API. This API is an modification version of ndn-cxx library and requires NFD running to forward interests. To compact with Consumer / Producer API and NFD, the project is also written in C++. We use Gstreamer 1.4.3 (other branch not tested) to process media. The supporting platform is UNIX-Like such as Mac OS and Linux. We will explain the implementation details about NDNLive and NDNTube respectively.

\subsection{NDNLive}
As we describe above (Figure~\ref{fig:ndnlive_arch}), the whole implementation is divided into producer host and consumer host. We will introduce the implementation details of each side, then describe some other vital parts we should pay attention to, such as \textit{Signing and Verification, Synchronization}. 
\begin{figure*}%[htbp]
  \centering
  \includegraphics[scale=0.3]{ndnlive_naming_pro}
  % \vspace{-0.3cm}
  \caption{NDNLive Producer and Consumer Structure}
  \label{fig:ndnlive_cp}
  %\vspace{-0.2cm}
\end{figure*}

\subsubsection{Producer}
\label{ssub:ndnlive_pro}
Four producers are presented in producer host (Figure~\ref{fig:ndnlive_cp}): video content producer, video stream information producer, audio content producer and audio stream information producer. 

The content\_producer keeps producing frames with frame number increasing incrementally (Figure~\ref{fig:ndnlive_naming})and publish them to the NDN Network. 

The stream\_info producer aims to provide the information about the live streaming such as frame rate, width, height, stream format. What's more, to help the consumer to catch up with the latest frame, the current frame number should also be included. To distinguish the obsolete stream\_info, timestamp will be appended at the end of name (Figure~\ref{fig:ndnlive_naming}).

\paragraph{Negative Acknowledgement} % (fold)
\label{par:negative_acknowledgement}
There are two situations we should consider carefully. 
\begin{enumerate}
	\item The first one is that, because once the consumer started consuming frames, it will have no idea the about the current frame number which producer is producing. The consumer may sometimes request for a frame number ahead of the producing. It is the producer's duty to inform the consumer about such knowledge. We introduce \textbf{NACK}(\textit{Negative Acknowledgment}) to handle such situation. 

	For example, in Algorithm~\ref{alg:liveproducer} , when the interest asks for a piece of data not existed (out of date or not be produced yet), this will trigger the \textit{cache\_miss} callback function (\textit{Process\_Interest}). In that function, if the data was not produced (\textit{not\_ready}), the producer will set up an \textit{APPLICATION\_NACK} with \textit{PRODUCER\_DELAY} option for this interest together with the estimated delay time.

	\item At the same time, although before the consumer starts to consume frames, it will ask for the current number, such information may also go out of date because of the network delay. These out-of-date frames will never be produced again, because the streaming is live. When faced with such situation, the producer will simply send a \textbf{NACK} with \textit{NO-DATA} option.
\end{enumerate}
% paragraph negative_acknowledgement (end)

\begin{algorithm}[ht]
\caption{NDNLive producer}
\label{alg:liveproducer}
\begin{algorithmic}[1]
\State $h_v \leftarrow $ \textbf{producer}(/ndn/ucla/ndnlive/stream-1/video/ \\\ content)
\State \textbf{setcontextopt}($h_v$, \textbf{cache\_miss}, \textit{ProcessInterest})
\State \textbf{attach}($h_v$)
\vspace{0.2cm}
	\While{\textit{TRUE}}
	\State $Name \textbf{ } suffix_v \leftarrow $ video frame number
	\State $content_v \leftarrow $ video frame captured from Camera
	\State \textbf{produce}($h_v$, $Name\textbf{ }suffix_v$, $content_v$)
	\EndWhile
\vspace{0.2cm}
\vspace{0.2cm}
\State $h_a \leftarrow $ \textbf{producer}(/ndn/ucla/ndnlive/stream-1/audio/ \\\  content)
\State \textbf{setcontextopt}($h_a$, \textbf{cache\_miss}, \textit{ProcessInterest})
\State \textbf{attach}($h_a$)
\vspace{0.2cm}
	\While{\textit{TRUE}}
	\State $Name \textbf{ } suffix_a \leftarrow $ audio frame number
	\State $content_a \leftarrow $ audio frame captured from mirophone
	\State \textbf{produce}($h_a$, $Name\textbf{ }suffix_a$, $content_a$)
	\EndWhile
\vspace{0.4cm}
\Function{ProcessInterest}{Producer \textbf{h}, Interest \textbf{i}}
  \If{\textit{NOT Ready}}
    \State $appNack \leftarrow $ \textbf{AppNack}($i$, \textbf{RETRY-AFTER})
    \State \textbf{setdelay}($appNack$, $estimated\_time$)
    \State \textbf{nack}($h$, $appNack$)
  \EndIf
   \If{\textit{Out of Date}}
    \State $appNack \leftarrow $ \textbf{AppNack}($i$, \textbf{NO-DATA})
    \State \textbf{nack}($h$, $appNack$)
  \EndIf
\EndFunction
\end{algorithmic}
\end{algorithm}

\begin{algorithm}[hbt]
\caption{NDNLive consumer}
\label{alg:liveconsumer}
\begin{algorithmic}[2]
\State $h_v \leftarrow $ \textbf{consumer}(/ndn/ucla/ndnlive//stream-1/video/\\\ content, \textit{UDR})
%\State \textbf{setcontextopt}($h_v$, \textit{EMBEDDED\_MANIFESTS}, \textit{TRUE})
%\State \textbf{setcontextopt}($h_v$, \textbf{receive\_buffer\_size}, 1MB)
\State \textbf{setcontextopt}($h_v$, \textbf{new\_segment}, \textit{ReassambleVideo})
\vspace{0.2cm}
	\While{\textit{reach Consume\_Interval\_Video}}
	\State $Name \textbf{ } suffix_v \leftarrow $ video frame number
	\State \textbf{consume}($h_v$, $Name\textbf{ }suffix_v$)
	\State $framenumber ++$
	\EndWhile
\vspace{0.2cm}

\Function{ReassambleVideo}{Data \textbf{segment}}
    \State $content \leftarrow $ reassamble \textbf{segment}
    \If{\textit{Final\_Segment}}
		\State $video \leftarrow $ decode \textbf{content}
	   	\State Play $video$
	\EndIf
\EndFunction

\vspace{0.4cm}

\State $h_a \leftarrow $ \textbf{consumer}(/ndn/ucla/ndnlive/stream-1/audio/\\\ content, \textit{SDR})
\State \textbf{setcontextopt}($h_a$, \textbf{new\_content}, \textit{ProcessAudio})
\vspace{0.2cm}
	\While{\textit{reach Consume\_Interval\_Audio}}
	\State $Name \textbf{ } suffix_a \leftarrow $ audio frame number
	\State \textbf{consume}($h_a$, $Name\textbf{ }suffix_a$)
	\State $framenumber ++$
	\EndWhile
\vspace{0.2cm}

\Function{ReassambleAudio}{Data \textbf{content}}
%   \State $video \leftarrow $ decode \textbf{content}
   	\State $audio \leftarrow $ decode \textbf{content}
   	\State Play $audio$
\EndFunction
\end{algorithmic}
\end{algorithm}

\subsubsection{Consumer}
\label{ssub:ndnlive_con}
Before the consumer asks for the true video data, it must fetch the live stream information to set up the Gstreamer playing pipeline. There are four consumers: video content consumer, video stream information consumer, audio content consumer and audio stream information consumer. 

\paragraph{Data Retrieval Protocol}
There are three Data Retrieval Protocols in Consumer / Producer API : \textbf{SDR, UDR, RDR}. We will illustrate which protocol we used for each consumer.
\begin{enumerate}
	\item {\textit{Content Retrieval}}
	
	Considering about the live streaming situation, the consumer part needs to keep the video and audio retrieving progress running all the time. The aim is to fetch all the segments inside one frame as soon as possible. The fetching process should NOT be blocked because of one segment missing. So we use \textbf{UDR} (\textit{Unreliable Data Retrieval}) for frames retrieval of living streaming. Because the \textbf{UDR} will pipeline the Interests sending, and the segments may received out of order, then the consumer part should take care of the segments reassemble and ordering stuff. If one segment of frame is not retrieved on time, then the whole frame will be skipped. But for audio, the size of audio frame is small enough to fill in one segment, so we would use \textbf{SDR} (\textit{Simple Data Retrieval}) for audio retrieval.

	\item {\textit{Stream Information Retrieval}} % (fold)
	
	Because the stream information contains only one segment and will be fetched only one time (at the beginning of the playing back). We use \textbf{SDR} (\textit{Simple Data Retrieval}) to fetch the stream info for video and audio. Except for the basic stream information, the consumer also needs to obtain the current frame number the producer just produced. So that the frame consumer can start from this frame number and increase it one by one. To retrieve the latest stream information, \textit{Right\_Most\_Child} option should be set as TRUE (Algorithm ~\ref{alg:liveconsumer}).
\end{enumerate}

\paragraph{Consume Interval} % (fold)
\label{par:consume_interval}
In consumer part, we should control the Interest sending speed. If we send them too aggressively, the data in producer side may not get ready. If we send them too slowly, the playing back may not match the video generating speed. Our solution is to send Interests according to the video and audio frame rate. For example, the video frame rate is 30 frame/second, then the \textit{consume\_interval} should be $1000/30 \approx {33.3}$ millisecond. The consume function should be called every \textit{consume\_interval}.
The boost scheduler will schedule the consume process every video or audio interval according to the video or audio \textit{consume\_interval}. 
% paragraph consume_interval (end)
\subsubsection{Some other vital parts}
\paragraph{Signing and Verification} % (fold)
\label{par:signing_and_verification}
Every NDN package should be signed with the producer's private key, only the verified frame can be retrieved successfully. But signing and verification are very time consuming. Consumer / Producer uses \textit{Manifest} \cite{api-tr} to improve the signing and verification performance. 

Instead of signing every segment in one frame, the producer only needs signing and verifying the Manifest. This option can be easily turned on or off by set \textit{EMBEDED\_\\\ MANIFEST} as TRUE or FALSE.
% paragraph signing_and_verification (end){Signing and Verification}

\paragraph{Synchronization between video and audio}
\label{par:sync}
Since we process video and audio separately, it is a vital problem to keep them synced. Gstreamer can handle the synchronization for us in this way:

When video and audio are captured, they are timestamped by the Gstreamer. The time information will be recorded in \textit{GstBuffer} data structure which Gstreamer used to contain media data. This time information will also be transferred along with video or audio frame. Then when the consumer fetches the video or audio frame separately, the video and audio frames will be pushed into the same \textit{GstQueue}. Gstreamer will extract the timestamps hiding in the video and audio frames, then play them back together according to the timestamps.

\subsection{NDNTube}
\begin{figure*}[ht]
  \centering
  \includegraphics[scale=0.3]{ndntube_naming_pro}
  % \vspace{-0.3cm}
  \caption{NDNTube Producer and Consumer Structure}
  \label{fig:ndntube_cp}
  %\vspace{-0.2cm}
\end{figure*}
Although NDNTube is very similar to NDNLive, data production and retrieval pattern are quite different from NDNLive. We will describe them in detail in this section.

\begin{algorithm}[ht]
\caption{NDNTube producer}
\label{alg:recordproducer}
\begin{algorithmic}[3]
\State $h_v \leftarrow $ \textbf{producer}(/ndn/ucla/ndntube/video-1234/ \\\ video)
\State \textbf{setcontextopt}($h_v$, \textbf{local\_repo}, \textit{TRUE})
\vspace{0.2cm}
	\While{\textit{NOT Final Frame}}
	\State $Name \textbf{ } suffix_v \leftarrow $ video frame number
	\State $content_v \leftarrow $ video frame
	\State \textbf{produce}($h_v$, $Name\textbf{ }suffix_v$, $content_v$)
	%\State $framenumber ++$
	\EndWhile
\vspace{0.2cm}
\vspace{0.2cm}
\State $h_a \leftarrow $ \textbf{producer}(/ndn/ucla/ndntube/video-1234/ \\\ audio)
\State \textbf{setcontextopt}($h_a$, \textbf{repo\_prefix}, \textit{/ndn/ucla/repo})
\vspace{0.2cm}
	\While{\textit{NOT EOF}}
	\State $Name \textbf{ } suffix_a \leftarrow $ audio frame number
	\State $content_a \leftarrow $ audio frame
	\State \textbf{produce}($h_a$, $Name\textbf{ }suffix_a$, $content_a$)
	%\State $framenumber ++$
	\EndWhile
\end{algorithmic}
\end{algorithm}

\begin{algorithm}[ht]
\caption{NDNTube consumer}
\label{alg:recordconsumer}
\begin{algorithmic}[4]
\State $h_v \leftarrow $ \textbf{consumer}(/ndn/ucla/ndntube/video-1234/ \\\ video, \textit{RDR})
%\State \textbf{setcontextopt}($h_v$, \textit{EMBEDDED\_MANIFESTS}, \textit{TRUE})
%\State \textbf{setcontextopt}($h_v$, \textbf{receive\_buffer\_size}, 1MB)
\State \textbf{setcontextopt}($h_v$, \textbf{new\_content}, \textit{ProcessVideo})
\vspace{0.2cm}
	\While{\textit{NOT EOS}}
	\State $Name \textbf{ } suffix_v \leftarrow $ video frame number
	\State \textbf{consume}($h_v$, $Name\textbf{ }suffix_v$)
	\State $framenumber ++$
	\EndWhile
\vspace{0.2cm}

\Function{ProcessVideo}{byte[] \textbf{content}}
   \State $video \leftarrow $ decode \textbf{content}
%   \State $audio \leftarrow $ decode \textbf{content}
   \State Play $video$
\EndFunction

\vspace{0.4cm}

\State $h_a \leftarrow $ \textbf{consumer}(ndn/ucla/ndntube/video-1234/ \\\ audio, \textit{RDR})
\State \textbf{setcontextopt}($h_a$, \textbf{new\_content}, \textit{ProcessAudio})
\vspace{0.2cm}
	\While{\textit{NOT Final Frame}}
	\State $Name \textbf{ } suffix_a \leftarrow $ audio frame number
	\State \textbf{consume}($h_a$, $Name\textbf{ }suffix_a$)
	\State $framenumber ++$
	\EndWhile
\vspace{0.2cm}

\Function{ProcessAudio}{byte[] \textbf{content}}
%   \State $video \leftarrow $ decode \textbf{content}
   	\State $audio \leftarrow $ decode \textbf{content}
   	\State Play $audio$
\EndFunction
\end{algorithmic}
\end{algorithm}

\subsubsection{Producer}

There are three producers: playlist producer, video producer and audio producer.
Playlist producer is responsible for generating the latest playlist every time it detected video file added or deleted.

Different from NDNLive, we combine content and stream\_info producer into one video or audio producer. Because for NDNLive, producers need to respond to the Interests coming from consumer directly. And content producer and stream\_info producer have different callbacks when the Interest enters the contexts or a cache\_miss is triggered, so we should separate them. But for NDNTube, all the stream\_info and content will be inserted into repo, and repo will take care of the response to consumers. There is no need to separate them. And once the producer finished producing the content and stream information, it can be offline, doesn't need to attach to the NDN Network. 

The producer side's Pseudocode is shown as Algorithm~\ref{alg:recordproducer}.

\subsubsection{Consumer}

There are five consumers: playlist consumer, video content consumer, video stream information consumer, audio content consumer and audio stream info consumer.

\paragraph{Data Retrieval Protocol} % (fold)
\label{par:ndntube_data_retrieval}

% paragraph data_retrieval (end)
Same with \textit{Streaminfo} tetrieval of NDNLive, we use \textbf{SDR} to retrieve stream information and the playing list. We want to fetch the latest version of playing list, so we should set \textit{Right\_Most\_Child} option as TRUE as well.

However, for the content retrieval, we should use \textbf{RDR} (\textit{Reliable Data Retrieval}). Because we can't stand any segment missing for the pre-recorded video, and we always want the good quality of video and audio. If the video segments are not received on time, the Interest requesting for that segment will be retransmitted. This retransmission is done by Consumer / Producer API. 

If all the retransmission failed to get the data. The consumer will resend the Interest for that frame. Such retransmission is application level. It won't send the Interest asking for the next frame until it gets the requested frame or several times application level retransmission. At the same time, when lacking of frames the \textit{Buffering} mechanism will be triggered. Only when Gstreamer accumulates enough video and audio frames (such as two seconds duration), it will continue to play back. Otherwise, it will be just paused until the buffer is full.

\paragraph{Other issues} % (fold)
\label{par:synchronization_between_video_and_audio}
There also exists the synchronization problem between video and audio. As we describe above~\ref{par:sync}, the Gstreamer will handle the synchronization part as long as we give the video and audio frame correct timestamps. In NDNLive, it is the capturing component who stamps the frames. In NDNTube, it is the \textit{Dumxer} who is responsible for time stamping. Once the media data flows through \textit{Dumxer}, this component will separate the video stream and audio stream according to their file type such as \textit{MP4} and adding the time information in each \textit{GstBuffer}.

Because all the content and stream information are all already existed and written into Repo. Then Repo takes over the responsibility. There are not \textit{NACK} in producer part. Also due to this reason, the consumer side should retrieve the data as soon as possible to keep the quality and fluency of video playing back. The default \textit{Consume\_Interval} is 0 in NDNTube.  

The consumer side's Pseudocode is shown as Algorithm~\ref{alg:recordconsumer}.
% paragraph synchronization_between_video_and_audio (end)

% section implementation (end)
%!TEX root = nextndnvideo-tr.tex
\section{prior work} % (fold)
\label{sec:comparison}
\begin{figure}%[htbp]
  \centering
  \includegraphics[scale=0.3]{ndnvideo_naming}
  % \vspace{-0.3cm}
  \caption{Prior NDNVideo Naming Space}
  \label{fig:ndnvideo_naming}
  %\vspace{-0.2cm}
\end{figure}
An similar work called NDNVideo was described in this technical report~\cite{ndnvideo}. Their aims are also to provide live and pre-recorded video streaming over NDN. They use Gstreamer to process media and Repo as the permanent storage. Producer and consumer concept are also the same, too. 

But the way how we handle framing are quite different. In the prior NDNVideo project, the video or audio stream is chopped into fixed size (segmentation) . A mapping between time and segment number is introduced to keep the video and audio synced (Figure~\ref{fig:ndnvideo_naming}). The seeking is also supported by the time-segment mapping mechanism. 

In our project, the video and audio is chopped into frames. One frame may contain several segments. The segmentation process is handled by Consumer / Producer API. The application only focuses on the frame level and leave other task to Consumer / Producer API. We think this application level framing is more like the true NDN way, which we mentioned in Section~\ref{sec:background}. Every frame has a unique name and is produced and consumed in one time. Only one frame missing won't affect other frames, thus leverage the whole impact to the playing back. 

On the contrary, the fixed size segmentation breaks the integrity of frames (ADU boundary). Only when all the packages are received correctly, the playing back progress can be guaranteed. So we think the prior NDNVideo is more like a TCP way. The application level framing also provides the flexibility to the video consumer. For example, in NDNLive, if the previous frame can't be retrieved on time or not integrated, the consumer can just skip this bad frame to keep the video streaming. We can see from the evaluation that it won't influence the video fluency. Table~\ref{table:comparison} shows other differences such as dependencies, Gstreamer version and coding language.

\begin{table}[ht]	
	\begin{tabular}{|c|c|c|}
	\hline
	             & NDNLive \& NDNTube                                                              & NDNVideo                                                     \\ \hline 
	Dependencies & \begin{tabular}[c]{@{}c@{}}ndn-cxx / NFD\\ Consumer / Producer API\end{tabular} & \begin{tabular}[c]{@{}c@{}}CCNx / CCNR \\ pyccn\end{tabular} \\ \hline
	Gstreamer    & 1.x                                                                             & 0.1                                                          \\ \hline
	Framing      & video \& audio frames                                                           & fixed segments                                               \\ \hline
	Language     & c++                                                                             & python                                                       \\ \hline
	\end{tabular}
	\caption{Comparison with NDNVideo}
	\label{table:comparison}
\end{table}
% section comparison (end)
%%!TEX root = nextndnvideo-tr.tex
\section{evaluation} % (fold)
\label{sec:evaluation}
When some frames missing, the performance does not be affected too much. The audio will not be affected at all. For video, when the missing video frame is the key frame, it will appear one second mosaic. But if it was other type frame, the picture is fluent enough.

[I really don't know what to talk about this part... All evaluation result seems not pretty good...] -- How to measure the quantification?

% section evaluation (end)
%!TEX root = nextndnvideo-tr.tex
\section{conclusion} % (fold)
\label{sec:conclusion}
In this technical report, we have proposed two video streaming project based on NDN, which are NDNLive and NDNTube. They all follow the ADU design pattern by chopping the stream into video or audio frames but not the fixed-size segmentation, which will not preserve the Application Data Units' boundaries. This provides the flexibility and also efficiency to the application. We introduce the architecture, name space and implementation details of them respectively. We consider they are the first and complete implementation by using a new NDN Interests-Data exchanging library called Consumer / Producer API. NDNLive and NDNTube have shown some basic usages of that API and can also be treated as a good tutorial of how to use Consumer / Producer API.
% section section_name (end)

%\input{10-acknowledgement}

\bibliographystyle{IEEEtran}
\bibliography{references}


\end{document}
