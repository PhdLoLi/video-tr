%!TEX root = nextndnvideo-tr.tex
\section{introduction} % (fold)
\label{sec:intro}
We have seen great changes of Internet communication pattern in recent years. The Named Data Networking (NDN) was proposed as a new Internet architecture that aims to overcome the weaknesses of the current host-based communication architecture in order to naturally accommodate emerging patterns of communication \cite{ndn-conext,ndn-tr,NDN-CCR14}. By naming data instead of their locations, NDN transforms data into a first class entity, which offers significant promise for content distribution applications, such as video playback application. NDN reduces network traffic by enabling routers to cache data packets. If multiple users request the same video file, the router can forward the same packet to them instead of requiring the video publisher to generate a separate packet. On the contrary, in current TCP/IP implementation, when clients request the same video, the publisher needs to send duplicate packets to transfer the exactly same video. 

What's more, in NDN consumers send Interest packets carrying application level names to request information objects, and the network returns the requested Data packets following the path of the Interests. The naming strategy greatly enables the flexibility of application designing. Because applications work with Application Data Units (ADU) --- units of information represented in a most suitable form for each given use-case~\cite{Clark1990}. For example, a multi-user game's ADUs are objects representing current user's status; for an intelligent home application, ADUs represent current sensor readings; and for a video playback application, data is typically handled in the unit of video frames. The naming space just matches the ADU naturally. 

However, we found that ADUs are not well considered in traditional video playback application running over TCP/IP. For example, MPEG-DASH technique~\cite{stockhammer2011dynamic} works by breaking multiplexed or unmultiplexed content into a sequence of small file segments of equal time duration. File segments are later served over HTTP from the origin media servers or intermediate HTTP caching servers. And such segmentation does not preserve boundaries of video frames (ADUs). But in our project every video or audio frame has a unique name, NDN segmentation exposes these boundaries through naming. These frames can be fetched independently according to user's different needs. For example, Consumer application can skip some video frames when packet losses occur in order to keep playing the actual `live' video. 

In this technical report, we will propose two NDN-based video project: NDNLive and NDNTube. They are live and pre-recorded video streaming project over NDN, which follows the ADU designing pattern. The following sections are organized as below. We will introduce Consumer / Producer API~\cite{api-tr} and Gstreamer~\cite{gstreamer}, which are the libraries we use for NDN Interests-Data exchanging and media processing in Section~\ref{sec:background}. The prior work will be compared in Section~\ref{sec:comparison}. Then we talk about the architecture and implementation of each project in Sections~\ref{sec:design} and ~\ref{sec:implementation}. Some experimental results will be shown in Section~\ref{sec:evaluation}. At last, we will conclude our projects in Section~\ref{sec:conclusion}.

% section section_name (end)