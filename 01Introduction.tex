%!TEX root = nextndnvideo-tr.tex
\section{introduction} % (fold)
\label{sec:intro}
The Named Data Networking (NDN) was proposed as a new Internet architecture that aims to overcome the weaknesses of the current host-based communication architecture in order to naturally accommodate emerging patterns of communication \cite{ndn-conext,ndn-tr,NDN-CCR14}. By naming data instead of their locations, NDN transforms data into a first class entity, which offers significant promise for content distribution applications, such as video playback application. NDN reduces network traffic by enabling routers to cache data packets. If multiple users request the same video file, the router can forward the same packet to them instead of requiring the video publisher to generate a new packet every time. 

NDN applications work with Application Data Units (ADU) --- units of information represented in a most suitable form for each given use-case (e.g. Application Level Framing~\cite{Clark1990}). For example, a multi-user game's ADUs are objects representing current user's status; for an intelligent home application, ADUs represent current sensor readings; and for a video playback application, data is typically handled in the unit of video frames. 

NDN publisher applications publishes ADUs as a series of Data packets according to the design of the namespace. NDN consumer applications send Interest packets carrying application level names to request ADUs, and the network returns the requested Data packets following the path of the Interests. 

Some of the existing techniques, such as MPEG-DASH partly adopt the principles of Application Level Framing by breaking multiplexed or unmultiplexed content into a sequence of small file segments of equal time duration~\cite{stockhammer2011dynamic}. File segments are later served over HTTP from the origin media servers or intermediate HTTP caching servers. 

Since the file segments have a bigger granularity than individual video and audio frames, the user of MPEG-DASH video streaming application still experiences interruptions and abrupt changes in the media quality. 
 
In this technical report, we talk about two video streaming NDN applications: NDNlive and NDNtube. NDNlive is capable of streaming live video captured by the camera and handling network problems by dropping individual video or audio frames. NDNtube prototypes the user experience of Youtube by providing dynamically generated playlist and streaming the media in its original quality (no frames are dropped). Both applications were developed on top of Consumer / Producer API~\cite{api-tr} providing a convenient way of publishing and fetching ADUs of any size, and Gstreamer~\cite{gstreamer} library providing media decoding/encoding functionality (Section~\ref{sec:background}). Architecture and implementation details of both applications are described in Section \ref{sec:arch} and \ref{sec:implementation}. The prior work is discussed in Section~\ref{sec:comparison}, followed by the conclusion in Section~\ref{sec:conclusion}.