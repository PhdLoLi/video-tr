%!TEX root = nextndnvideo-tr.tex
\section{comparison to prior NDNVideo} % (fold)
\label{sec:comparison}
The main difference between Next-NDNVideo and NDNVideo is the way how we handle framing. In the prior NDNVideo project, the video or audio stream is chopped into fixed size. The segment can just fill in one NDN package. A mapping between time and segment number is introduced to keep the video and audio synced. The seeking is also supported by the time-segment mapping mechanism. 

In our project, the video and audio is chopped into frames. One frame may contain several segments. The segmentation process is handled by Consumer / Producer API. The application only focuses on the frame level and leave other task to Consumer / Producer API. We think this application level framing is more like the true NDN way, which we mentioned in chapter 2\ref{sec:background}. Every frame has a unique name and is produced and consumed in one time. Only one frame missing won't affect other frames, thus leverage the whole impact to the playing back. 

On the contrary, the fixed size segmentation breaks the integrity of frames. Only when all the packages are received correctly, the playing back progress can be guaranteed.  So we think the prior NDNVideo is more like a TCP way, but not NDN.  The application level framing also provides the flexibility to the video consumer. For example, in the live streaming case, if the previous frame can't be retrieved on time or not integrated, the consumer can just skip this bad frame to keep the video streaming. I can see that it won't influence the video fluency according to our evaluation. Table one shows other differences such as dependencies, Gstreamer version and coding language.

[table needed here]
% section comparison (end)